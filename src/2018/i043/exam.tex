\documentclass[a4paper,12pt,reqno]{amsart}

\usepackage{graphicx}

\usepackage{amssymb}
\usepackage{amsthm}
\theoremstyle{definition}

\usepackage{tikz}
\usetikzlibrary{plotmarks}
\usetikzlibrary{patterns}
\usetikzlibrary{decorations.markings}
\usetikzlibrary{math}

\newtheorem*{theorem*}{Theorem}
%% this allows for theorems which are not automatically numbered

\newtheorem{definition}{Definition}
\newtheorem{theorem}{Theorem}
\newtheorem{lemma}{Lemma}
\newtheorem{example}{Example}
\newtheorem{problem}{Problem}
\newtheorem{quest}{Question}
\usepackage{lineno}

\usepackage[shortlabels]{enumitem}

%
%
%

\title{The Second Trial - Year 2018}

\author{\emph{Esame di Maturit\'a} translated by Andrea Tino}

\begin{document}

\begin{abstract}
English translation from the regular session of the Italian National Exam
(LI02, EA02, LI03 and LI15), Mathematics, year 2018.
\end{abstract}

\maketitle

The student is required to solve one out of the two problems, and to answer 5 questions. 

\section{Problems}

\begin{problem}
You are required to program a machine employed in a factory for the production of floor tiles.
Tiles are square-shaped of side $1$ (in a suitable unit of measure), the production stages are the following:
\begin{enumerate}[(a)]
\item Function $y=f(x)$ is chosen, defined in interval $[0,1]$ and therein continuous; which meets the following
conditions:
\begin{itemize}
\item \label{itm:1} $f(0) = 1$
\item \label{itm:2} $f(1) = 0$
\item \label{itm:3} $0 < f(x) < 1$ for $0 < x < 1$.
\end{itemize}
\item The machine draws the plot $\Gamma$ of function $y=f(x)$ and the symmentric plots of $\Gamma$
to axis $y$, axis $x$ and origin $O$, obtaining therefore a closed curve $\Lambda$ crossing points
$(1,0)$, $(0,1)$, $(-1,0)$, $(0,-1)$, symmetric to the axes and the origin, contained inside square $Q$
with vertices $(1,1)$, $(-1,1)$, $(-1,-1)$, $(1,-1)$.
\item The machine builds one tile by colouring grey the inner area inside curve $\Lambda$ and leaving blank
the remaining surface in square $Q$. On the screen, are then shown some tiles side by side, to provide an idea
of the final look of the floor.
\end{enumerate}
The manual shows an example of the realization process of a simple tile (figure \ref{fig:fshape}). The
final result is also shown in the manual as per figure x.
%
% Figure
%-----------------------------------------------------------
   \begin{figure}
   \centering
\tikzstyle{smallfig}=[scale=0.25]
   %
\begin{tikzpicture}[style=smallfig]
\draw[thin,->] (-6,0) -- (6,0) node[right] {$\hat{x}$};
\draw[thin,->] (0,-6) -- (0,6) node[above] {$\hat{y}$};

\fill (0,4) circle (5pt) node[above left] {$1$};
\fill (4,0) circle (5pt) node[below right] {$1$};
\fill (-4,0) circle (5pt) node[below left] {$-1$};
\fill (0,-4) circle (5pt) node[below left] {$-1$};

\draw[thick] (0,4) -- (4,0);
\end{tikzpicture}
\quad
\begin{tikzpicture}[style=smallfig]
\draw[thin,->] (-6,0) -- (6,0) node[right] {$\hat{x}$};
\draw[thin,->] (0,-6) -- (0,6) node[above] {$\hat{y}$};

\fill (0,4) circle (5pt) node[above left] {$1$};
\fill (4,0) circle (5pt) node[below right] {$1$};
\fill (-4,0) circle (5pt) node[below left] {$-1$};
\fill (0,-4) circle (5pt) node[below left] {$-1$};

\draw[thick] (0,4) -- (4,0);
\draw[thick] (-4,0) -- (0,4);
\draw[thick] (0,-4) -- (-4,0);
\draw[thick] (0,-4) -- (4,0);
\end{tikzpicture}
\quad
\begin{tikzpicture}[style=smallfig]
\path[fill=lightgray!40] (4,0) -- (0,4) -- (-4,0) -- (0,-4);

\draw[thin,->] (-6,0) -- (6,0) node[right] {$\hat{x}$};
\draw[thin,->] (0,-6) -- (0,6) node[above] {$\hat{y}$};

\fill (0,4) circle (5pt) node[above left] {$1$};
\fill (4,0) circle (5pt) node[below right] {$1$};
\fill (-4,0) circle (5pt) node[below left] {$-1$};
\fill (0,-4) circle (5pt) node[below left] {$-1$};

\draw[thick] (0,4) -- (4,0);
\draw[thick] (-4,0) -- (0,4);
\draw[thick] (0,-4) -- (-4,0);
\draw[thick] (0,-4) -- (4,0);

\draw (4,4) -- (-4,4);
\draw (-4,4) -- (-4,-4);
\draw (-4,-4) -- (4,-4);
\draw (4,-4) -- (4,4);
\end{tikzpicture}
   %
   \caption{The three stages of production (from left to right) of a simple tile.}
   \label{fig:fshape}
   \end{figure}
%-----------------------------------------------------------
%
\begin{enumerate}
\item In relation to the example, determine the expression of function $y=f(x)$ and the equation of curve
$\Lambda$, in order to verify the correct machine's functioning.
\end{enumerate}
You are asked to build a tile with a more elaborate decoration which, other than satisfying conditions \ref{itm:1},
\ref{itm:2} and \ref{itm:3} introduced before, is such that $f^\prime(0) = 0$ and the coloured area occupy
55 \% of the total tile's surface. On this regard, take into consideration polynomial functions of 2\textsuperscript{nd}
and 3\textsuperscript{rd} degree.
\begin{enumerate}[resume]
\item After assessing that it is not possible to realize what's required by employing a polynomial function of
2\textsuperscript{nd} degree, determine coefficients $a,b,c,d \in \mathbb{R}$ of 
3\textsuperscript{rd} degree polynomial function $f(x)$ which satisfies the requirements. In the end, plot in
a cartesian diagram the final tile.
\end{enumerate}
The client is offered two types of decorations, each respectively coming from functions $a_n(x)=1-x^n$
and $b_n(x)=(1-x)^n$, both considered in $x\in[0,1]$ with $n$ being a positive integer.
\begin{enumerate}[resume]
\item Verify that, on varying of $n$, all these functions meet conditions \ref{itm:1},
\ref{itm:2} and \ref{itm:3}. Said $A(n)$ and $B(n)$ be the areas of the coloured parts of the tiles by
using functions $a_n$ and $b_n$, calculate $\lim\limits_{n \to +\infty} A(n)$ and 
$\lim\limits_{n \to +\infty} B(n)$, and
process the results from a geometric point of view.
\end{enumerate}
The client decides to order 5.000 tiles with decoration derived from $a_2(x)$ and 5.000 with the one derived from
$b_2(x)$. Painting is performed by an automated mechanical arm which, after laying the color, returns to its
initial position hovering the tile along its diagonal. Due to a malfunctioning, during the poroduction of the 10.000
tiles, it occurs with a probability 20\% that the mechanical arm leaves a drop of colour on a casual point 
along the diagonal, such that the produced tile gets dirty.
\begin{enumerate}[resume]
\item Provide a well motivated estimation of the number of tiles which, having a dirty spot on the non-coloured
part, will be regarded as demaged at the end of the production cycle.
\end{enumerate}
\end{problem}

\begin{problem}
Consider function $f_k: \mathbb{R} \mapsto \mathbb{R}$ defined as:
\begin{equation}
f_k(x)=-x^3+kx+9
\end{equation}
With $k\in\mathbb{Z}$.
\begin{enumerate}
\item Let $\Gamma_k$ be the function's plot, verify that, for every value of parameter $k$, line $r_k$ tangent
$\Gamma_k$ on point of x-coordinate $0$ and line $s_k$ tangent $\Gamma_k$ on point of x-coordinate $1$ cross
in point $M$ of x-coordinate $\frac{2}{3}$.
\item After assessing that $k=1$ is the maximum positive integer such that the y-coordinate of point $M$ is
lower than $10$, analyze the behavior of function $f_1(x)$, determining its stationary and inflection points by
drawing its graph.
\item Let $T$ be the triangle delimited by lines $r_1$, $s_1$ and the x-axis; calculate the probability that,
for every random point $P(x_p,y_p)$ inside $T$, it is located above $\Gamma_1$ (that is $y_p > f_1(x)$ in said
point $P$).
\item Figure x shows a point $N \in \Gamma_1$ and a segment of graph $\Gamma_1$. The line normal to
$\Gamma_1$ in $N$ (that is the perpendicular to the line tangent $\Gamma_1$ in such point) crosses the
origin $O$. Graph $\Gamma_1$ includes three point with such property. Prove, more generally, that the graph
of whatever polynomial of degree $n>0$ cannot include more than $2n-1$ points in which the normal line to the
graph crosses the origin.
\end{enumerate}
\end{problem}

\section{Questionnaire}

\begin{quest}
Prove that the volume of a cylinder inscribed inside a cone is lower than half of the cone's volume.
\end{quest}

\begin{quest}
For any set $A$, $\emptyset \subset A$.
\end{quest}

\begin{quest}
For any set $A$, $\emptyset \subset A$.
\end{quest}

\begin{quest}
For any set $A$, $\emptyset \subset A$.
\end{quest}

\begin{quest}
For any set $A$, $\emptyset \subset A$.
\end{quest}

\begin{quest}
For any set $A$, $\emptyset \subset A$.
\end{quest}

\begin{quest}
For any set $A$, $\emptyset \subset A$.
\end{quest}

\begin{quest}
For any set $A$, $\emptyset \subset A$.
\end{quest}

\begin{quest}
For any set $A$, $\emptyset \subset A$.
\end{quest}

\begin{quest}
For any set $A$, $\emptyset \subset A$.
\end{quest}

\end{document}