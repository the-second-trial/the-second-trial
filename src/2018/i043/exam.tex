\documentclass[a4paper,12pt,reqno]{amsart}

\usepackage{graphicx}

\usepackage{amssymb}
\usepackage{amsthm}
%\theoremstyle{plain}

\newtheorem*{theorem*}{Theorem}
%% this allows for theorems which are not automatically numbered

\newtheorem{definition}{Definition}
\newtheorem{theorem}{Theorem}
\newtheorem{lemma}{Lemma}
\newtheorem{example}{Example}
\newtheorem{problem}{Problem}
\newtheorem{quest}{Question}
\usepackage{lineno}

%% The above lines are for formatting.  In general, you will not want to change these.


\title{Esame di Maturit\'a - Year 2018}

\author{translated by Andrea Tino}

\begin{document}

\begin{abstract}
English translation from the regular session of the National Exam, Mathematics, year 2018.
\end{abstract}

\maketitle

The student is required to solve one out of the two problems, and to answer 5 questions. 

\section{Problems}

\begin{problem}
You are required to program a machine employed in a factory for the production of floor tiles.
Tiles are square-shaped of side $1$ (in a suitable unit of measure), the production stages are the following:
\begin{enumerate}
\item Function $y=f(x)$ is chosen, defined in interval $[0,1]$ and therein continuous; which meets the following
conditions:
\begin{itemize}
\item $f(0) = 1$
\item $f(1) = 0$
\item $0 < f(x) < 1$ for $0 < x < 1$.
\end{itemize}
\item The machine
\item sdsd
\end{enumerate}
dfdf
\end{problem}

Notice that for sets $A$ and $B$, if $A \not\subset B$, then there exists an element $x$ such that $x \in A$ and $x \notin B$.  That is,
$$(A \not\subset B) \leftrightarrow \exists x ( x \in A \wedge x \notin B).$$

\begin{problem}
Let $A = \{ 1, 2, 3, 4, 5 \} $, $B = \{ 1, 2 \}$ and $C = \{ 1, 7 \}$.  We can see that every element in $B$ is an element of $A$.  Further, we can see that $C$ contains an element, namely $7$, which is not in $A$.  Thus, $B \subset A$ and $C \not\subset A.$
\end{problem}

We now prove theorem $2.1$.

\begin{theorem*}[2.1]
For any set $A$, $\emptyset \subset A$.
\end{theorem*}

\begin{proof}
By way of contradiction, suppose that the theorem fails.  Let $A$ be a set such that $\emptyset \not\subset A.$  From the above discussion, we can see that there exists an element $x$ such that $x \in \emptyset$ and $x \notin A$.  Let $x$ be such an element.  Since the emptyset has no elements, then $x \notin \emptyset$.  Thus, we have that $x \in \emptyset$ and $x \notin \emptyset.$ This contradiction proves that the theorem is true.
\end{proof}



\end{document}