\section{Introduction}

The first problem, as usual, offers an applicative context very common and important
nowadays in the field of \emph{Automation Factory}: here mathematical techniques
can be used to automate certain processes. The problem also integrates elements of
Probability Theory to provide a full real case analysis around process failure and how
to efficiently fix issues in a production line. The mathematical skills required to solve
this problem revolve around basic calculus, function symmetry, polynomials and
basic concepts of probability.

The second problem is more theoretical without any applicative scenarios. The
mathematical skills required to solve it include polynomials and basic calculus.

The questionnaire spans across different themes ranging from basic equations,
simple integration, Geometry, Calculus, Probability and differential equations.

The final assessment on the overall exam sets its difficulty to an average level.