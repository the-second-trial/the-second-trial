\newcommand{\snippetrationalroottheorem}{%
    The \textit{Rational Root Theorem} states that
    if an $n$-order polynomial $P(x)$:
    \begin{equation*}
        P(x) = a_x x^n + a_{n-1}x^{n-1} + \dots + a_0
    \end{equation*}
    With all integer coefficients: $a_i \in \mathbb{N}, i = 0 \dots n$,
    has a rational root $x^\ast$:
    \begin{equation*}
        P(x^\ast) = 0 \wedge x^\ast = \frac{p}{q} \wedge p,q \in \mathbb{N}
    \end{equation*}
    Then that root can be found inside $\Omega_P$:
    \begin{equation*}
        \Omega_P \equiv \left\{ \frac{p_i}{q_j} : p_i \in \text{m}(a_0) \wedge q_j \in \text{m}(a_n) \right\}
    \end{equation*}
    Where $\text{m}(\alpha)$ indicates all the prime factors of integer $\alpha$
    (including $1$), all taken with both signs.    
}
