%%%%%%%%%%%%%%%%%%%%%%%%%%%%%%%%%%%%%%%%%%%%%%%%%%%%%%%%%%%%%%%%%%%%%%%%%%%%%%%
% The 2nd Trial by Andrea Tino is licensed under CC BY-NC-SA 4.0. To view a   %
% copy of this license, visit                                                 %
% http://creativecommons.org/licenses/by-nc-sa/4.0/                           %
%%%%%%%%%%%%%%%%%%%%%%%%%%%%%%%%%%%%%%%%%%%%%%%%%%%%%%%%%%%%%%%%%%%%%%%%%%%%%%%

\chapter[Magnetism and capacitors]{Magnetic field inside a
capacitor}
\label{ch:tufte-book}
\marginnote{Appears as \textit{second problem} in:
LI02, EA02 – Scientific; LI03 - Scientific - Applied Sciences;
LI15 - Scientific - Sport Sciences (standard session)}

This problem introduces a flat capacitor to which
a specific voltage is applied. The voltage (which is null at $t = 0$)
varies over time and it causes a magnetic field $\vec{B}$ to originate,
whose module\sidenote{Disregarding border effects.} is:
\begin{equation}\label{eq:bdef}
    B = \frac{k t}{\sqrt{\left( t^2 + a^2 \right)^3}} \cdot r
\end{equation}
Having the following
quantities\sidenote[][0.17\textwidth]{The text does not mention what MU is
assigned to $r$, but it is sensible to
infer that it should be the same as $R$ because a constraint is present
between them: $r \leq R \implies r - R \leq 0$, and since they sum together,
they must have the same MU.}:

\begin{table}[h]
\begin{center}
    \footnotesize%
    \begin{tabular}{cccl}
    \toprule
    Quantity & UM & Constraints & Description \\
    \midrule
    $\vec{B}$ & \unit{T} & & Magnetic field \\
    $R$ & \unit{m} & $R > 0$ & Capacitors plate's radius \\
    $d$ & \unit{m} & $d > 0$ & Distance between plates \\
    $r$ & \unit{m} & $r > 0 \wedge r \leq R$ & Distance from axis \\
    $k$ & & $k > 0$ & Constant \\
    $a$ & & $a > 0$ & Constant \\
    $t$ & \unit{s} & $t \geq 0$ & Time \\
    \bottomrule
    \end{tabular}
\end{center}
\caption{Most important quantities in the problem.}
\label{tab:quants}
\end{table}

\begin{figure}
    \centering
    \input{problem_1_L_overview.diag}
    \caption[Problem representation.][0pt]{Problem representation: a 3D view and a 2D
    simplified diagram.}
    \label{fig:capacitorovervw}
\end{figure}

\paragraph{What is the UM of $a$ and $k$?}
\marginnote{\smallcaps{Problem from text}}
We start off with a question
about \smallcaps{dimensionality}. By looking at equation \ref{eq:bdef}, we can
easily set up a dimensional equation:
\begin{equation*}
    \left[ B \right] = \left[ \frac{k t}{\sqrt{\left( t^2 + a^2 \right)^3}} r \right] \implies
    \unit{T} = \frac{[k] \cdot \unit{s} \cdot \unit{m}}{\sqrt{\left( \unit{s}^2 + [a]^2 \right)^3}}
\end{equation*}
Notice that, at denominator on the LHS, we have: $s^2 + [a]^2$. Since $[a]^2$
appears in a sum with $s^2$, it must have the same UM: $[a]^2 = s^2$.
That means $[a] = \unit{s}$, and the whole sum collapses into a single term: $s^2$:
\begin{equation*}
    \unit{T} = \frac{[k] \cdot \unit{s} \cdot \unit{m}}{\sqrt{\left( \unit{s}^2 \right)^3}} = 
    \frac{[k] \cdot \unit{s} \cdot \unit{m}}{\sqrt{\left( \unit{s}^3 \right)^2}} = 
    \frac{[k] \cdot \unit{s} \cdot \unit{m}}{\unit{s}^3} =
    \frac{[k] \cdot \unit{m}}{\unit{s}^2}
\end{equation*}
Notice how we further simplified the expression by removing the square root and removing
$\unit{s}$ terms. At this point, we have two choices: either we remember that
$T=\unit{Kg} \cdot \unit{s}^{-1} \cdot \unit{C}^{-1}$, or we try to calculate it from
Maxwell's
equations\sidenote{You can use Maxwell's 4th equation as described at: \nameref{sec:maxwell}.}.
Either way, we end up with:
\begin{equation*}
    \frac{\unit{Kg}}{\unit{s} \cdot \unit{C}} = \frac{[k] \cdot \unit{m}}{\unit{s}^2} \implies
    \frac{\unit{Kg}}{\unit{C}} = \frac{[k] \cdot \unit{m}}{\unit{s}} \implies
    \frac{\unit{Kg} \cdot \unit{s}}{\unit{C} \cdot \unit{m}} = [k]
\end{equation*}
Which means that $[k] = \unit{Kg} \cdot \unit{s} \cdot \unit{C}^{-1} \cdot \unit{m}^{-1}$.

\paragraph{Why $B \neq 0$ in the capacitor even without magnets or currents?}
\marginnote{\smallcaps{Problem from text}}
The answer to this
is given by remembering that a magnetic field can originate from two possible sources:
\begin{itemize}
    \item Electrical currents.
    \item Variations of electrical fields.
\end{itemize}
If you remember Maxwell's 4th
equation\sidenote{See \nameref{sec:maxwell}.},
then you can actually have a confirmation of these
two facts. In our problem, there are no magnets and there is no electrical current because a
capacitor opens the circuit blocking any current from flowing. However a voltage is applied,
that means an electrical field originates between the plates; and the voltage varies over
time which means the electrical field varies too, and that causes a magnetic field to
arise as per second point above.

\paragraph{How do the directions of $\vec{E}$ and $\vec{B}$ relate together?}
\marginnote{\smallcaps{Problem from text}}
Here as well you should remember that inside an electromagnetic field, the electrical
field $\vec{E}$ and the magnetic field $\vec{B}$ are always perpendicular to each other.
If you didn't remember that, take Maxwell's either 3rd of 4th
equation\sidenote{The proof that $\vec{E}$ and $\vec{B}$ are perpendicular can be
found at: \nameref{sec:maxwell}.},
and you will be able to check it is true.
