%%%%%%%%%%%%%%%%%%%%%%%%%%%%%%%%%%%%%%%%%%%%%%%%%%%%%%%%%%%%%%%%%%%%%%%%%%%%%%%
% The 2nd Trial by Andrea Tino is licensed under CC BY-NC-SA 4.0. To view a   %
% copy of this license, visit                                                 %
% http://creativecommons.org/licenses/by-nc-sa/4.0/                           %
%%%%%%%%%%%%%%%%%%%%%%%%%%%%%%%%%%%%%%%%%%%%%%%%%%%%%%%%%%%%%%%%%%%%%%%%%%%%%%%

\chapter[Questions]{Questions}
\label{ch:qs}

In this year's exams we can see a nice variaty of topics in the questions
ranging from the main theme: Magnetism to Probability and Linear Algebra.

\section{Extrema finding}
\label{sec:extremafind}
\marginnote{Appears as problem 1 in L 06.}
Let $p(x)$ be a polynomial, the following function is provided:
\begin{equation*}
    f(x) = \frac{p(x)}{x^2 + d}
\end{equation*}
Where $d \in \mathbb{R}$. $f$'s plot intersects the x-axis at $x_1 = 0$ and
$x_2 = \frac{12}{5}$, and has these lines as asymptotes: $x=3$, $x=-3$ and
$y=5$. We need to calculate the points of relative maximum and minimum.

\paragraph{}
The first thing important noticing is that $f$ is the ratio between two
polynomials. How much do we know about these two polynomials? The
denominator is a second-degree polynomial, while we do not have much
information about the numerator. We need to know a few things more about
$p$. From what the problem tells us, we know that
$\lim_{x \to \infty} f(x) \neq \infty$. That is a very important piece
of data because we know that the ratio between two polynomials tends
to a finite value (as the variable tends to infinity) only when they
have the same
degree\sidenote{The full theorem actually states that:
\begin{equation*}
    \lim_{x \to \infty} \frac{N(x)}{D(x)} = l \neq \infty
        \iff n = m
\end{equation*}
Where $N(x)$ and $D(x)$ are two polynomials with
degrees $n$ and $m$ respectively. And we also have that:
\begin{equation}
    l = \frac{a_N}{a_D}
\end{equation}
Where $a_N$ id $N$'s highest power's coefficient, and
$a_D$ id $D$'s highest power's coefficient.}.
Thanks to it, we can rewrite $f$ as:
\begin{equation}\label{eq:fracinfty}
    f(x) = \frac{ax^2 + bx + c}{x^2 + d}
\end{equation}
Because of equation \ref{eq:fracinfty}, we have that:
\begin{equation*}
    \lim_{x \to \infty} f(x) = \lim_{x \to \infty} \frac{ax^2 + bx + c}{x^2 + d}
        = \frac{a}{1} = 5
        \implies a = 5
\end{equation*}
As we unveil the values of coefficients, we know more and more the final form
of $f$, and this is important because we will need to compute its derivative and
then study its sign: the less parameters in the derivative, the easier our job.
Now $f$ is:
\begin{equation*}
    f(x) = \frac{5x^2 + bx + c}{x^2 + d}
\end{equation*}
We still have 3 parameters.
Let's use the info about intersections with the x-axis to find the values of some
of those:
\begin{equation*}
    \begin{split}
        &\begin{cases}
            f(0) = 0\\
            f\left( \frac{12}{5} \right) = 0
        \end{cases}
        \implies
        \begin{cases}
            \frac{c}{d} = 0\\
            \frac{5\frac{12^2}{5^2} + b\frac{12}{5} + c}{\frac{12^2}{5^2} + d} = 0
        \end{cases}
        \implies
        \begin{cases}
            c = 0\\
            \frac{\frac{12^2}{5} + b\frac{12}{5}}{\frac{12^2}{5^2} + d} = 0
        \end{cases}
        \implies\\
        &\frac{12^2 + 12b}{5} = 0 \implies
        12 + b = 0 \implies
        b = -12
    \end{split}
\end{equation*}
Which means now $f$ can be written as:
\begin{equation*}
    f(x) = \frac{5x^2 - 12x}{x^2 + d}
\end{equation*}
We only have $d$ left to find out, and for that we can use the information about
the two vertical asymptotes. Remember that $f$ is a ratio, when the denominator
tends to 0, the whole ratio tends to infinity; it means that if we find the
values of $x$ for which $x^2 + d = 0$, we will find the exact points where
the asymptots are: $x^2 + d = 0 \implies x = \pm \sqrt{-d}$. At this point
it becomes clear that $d < 0$ because otherwise we would have no asymptotes as
we would need to extract the square root of a negative number which is not a
real number.
Also notice that $d \neq 0$, otherwise we would not have two different
asymptotes, but only one. Since we know that the two aymptotes are $x = \pm 3$,
we can conclude that: $\pm 3 = \sqrt{-d} \implies d = -9$, and:
\begin{equation*}
    f(x) = \frac{5x^2 - 12x}{x^2 - 9}
\end{equation*}
Now we can compute $f^\prime$:
\begin{equation*}
    f^\prime(x) = \frac{10x - 12}{x^2 - 9} - \frac{5x^2 - 12x}{(x^2 - 9)^2} 2x = \dots =
        6\frac{2x^2 - 15x + 18}{(x^2 - 9)^2}
\end{equation*}
We first need to find the zeros\sidenote{Remember that the zeros
of a fraction $\frac{N(x)}{D(X)}$ can be found by finding the zeros of the numerator $N(x)$.}
of $f^\prime$:
\begin{equation*}
    \begin{split}
        2x^2 - 15x + 18 = 0 &\implies x = \frac{-b \pm \sqrt{\Delta}}{2a} =
        \frac{15 \pm \sqrt{81}}{4} = \frac{15 \pm 9}{4}\\
        &\implies
        \begin{cases}
            x_1 = \frac{15 + 9}{4} = \frac{24}{4} = 6\\
            x_2 = \frac{15 - 9}{4} = \frac{6}{4} = \frac{3}{2}
        \end{cases}
    \end{split}
\end{equation*}
Switching to the corresponding inequality: $2x^2 - 15x + 18 > 0$, we see that
the leading coefficient is positive and the inequality's sign is $>$; as
already stated in \ref{sn:dice}, we have
that: $f^\prime(x) > 0 \iff x < \frac{3}{2} \vee x > 6$.
%
\begin{marginfigure}[-5.3\textwidth]
    \includegraphics[width=1\linewidth]{qs_1_fplot_diag.tikz}
    \caption{Construction of the final plot.}%
    \label{fig:pr1fplot}%
\end{marginfigure}
%
This confirms that $x_1$ and $x_2$ are extrema (since around them $f^\prime$ changes sign),
and gives us the final result: $x_1$ is a point of relative maximum, while $x_2$ is
a point of relative minimum.

\section{A big polynomial}
\label{sec:bigpoly}
\marginnote{Appears as problem 1 in L 06.}
Let $g(x)$ be the following polynomial:
\begin{equation*}
    g(x) = \sum_{n = 1}^{1010} x^{2n-1} = x + x^3 + x^5 + \dot + x^{2017} + x^{2019}
\end{equation*}
We must prove that $g$ has a single real root. We also need to compute:
\begin{equation}\label{eq:limfracpolyexp}
    \lim_{x \to +\infty} \frac{g(x)}{1.1^x}
\end{equation}

\paragraph{}
The first part of the problem is about proving the \textbf{existence} and \textbf{uniqueness}
of $g$'s real root. In mathematical terms, this is what we need to prove:
\marginnote{You might see logical operator $\exists!$ for the first time: it
simply means: \textit{"there exists only one..."}.}
\begin{equation*}
    \exists! x_0 \in \mathbb{R} : g(x_0) = 0
\end{equation*}
The proof is actually much easier than you can expect. $g$ looks very scary because it's big,
but if we pay attention, we can see that the polynomial is missing the contant term $x^0$.
So we can actually rewrite $g$ by extracting $x$ as:
\begin{equation*}
    g(x) = x \cdot \left(1 + x^2 + x^4 + \dot + x^{2016} + x^{2018} \right)
\end{equation*}
Look now, we have basically applied the following transformation:
\begin{equation*}
    g(x) = \sum_{n = 1}^{1010} x^{2n-1} = x \cdot \sum_{n = 1}^{1009} x^{2n} 
\end{equation*}
At this point it's easy to see there exists only a single real root to $g$:
\begin{equation}\label{eq:gsumprodzero}
    g(x) = 0 \implies x \cdot \sum_{n = 1}^{1009} x^{2n} = 0
\end{equation}
And that is $x_0 = 0$:
\begin{enumerate}
    \item As equation \ref{eq:gsumprodzero} shows, $g$ is the product of two factors: $g$ is
        null when either one of them is null. One of these terms is expression $x$:
        therefore $x = 0$ is a real root. This proves the existence of $x_0$.
    \item The second factor in $g$'s definition as per equation \ref{eq:gsumprodzero},
        is a complete polynomial where all its coefficients are positive and all powers
        are even\sidenote{Expression $2n$ for $n = 1, 2, 3, \dots$ describes an even number.}.
        This means that the sum we see as second factor is a sum of squares, and
        we know that a sum of squares is never null (for real values). This proves the uniqueness
        of $x_0$.
\end{enumerate}

\paragraph{}
As for the second part of the problem, the limit in equation \ref{eq:limfracpolyexp} involves
a ratio between a polynomial and an exponential term. It is easy to prove that ratio tends
to $0$ when $x \to +\infty$:
\begin{enumerate}
    \item We know that an exponential term $a^x$ tends to $+\infty$ as the variable tends
        to $+\infty$ when $a > 1$. Therefore $1.1^x \to +\infty$.
    \item We know that polynomials always tend to $\infty$ when the variable tends to $\infty$.
        In this case, as $x \to +\infty$, $g(x) \to +\infty$ because the leading coefficient
        is positive.
    \item The ratio $\frac{g(x)}{1.1^x}$ tends to $0$ as $x \to +\infty$ because the exponential
        appears as the denominator. An exponential grows to $+\infty$ with an incredible speed.
        A polynomial cannot match that speed. Therefore the denominator grows much more than
        the numerator causing the whole fraction to tend towards $0$.
\end{enumerate}
If we want to be very precise and provide a proof to the last point, we can apply De L\^{o}pital's
theorem\textsuperscript{\ref{sn:dehopital}}. Both functions are
differentiable\sidenote{Actually they are both $\mathcal{C}^\infty$: indefinitely differentiable.},
and they both tend towards $\infty$; so we get the indeterminate
form $\left[ \frac{\infty}{\infty} \right]$. As we apply the rule once, we notice that the
indeterminate form still remains but the polynomial has dropped its order by 1; so we re-apply
the rule (as many time as the original order of the polynomial: $1010$) until we get $0$:
\marginnote{Each arrow indicates an iteration where we take the derivative of the numerator and
the denominator. Also note that $p^{[N]}(x)$ indicates a generic $N$-degree polynomial.}
\begin{equation*}
    \frac{p^{[N]}(x)}{1.1^x} \rightarrow \frac{p^{[N-1]}(x)}{\log(1.1)1.1^x}
        \rightarrow \frac{p^{[N-2]}(x)}{\log^2(1.1)1.1^x} \rightarrow \dots \rightarrow
        \frac{0}{\log^N(1.1)1.1^x} = 0
\end{equation*}
If we run the above process for $N=1010$, we get the result.

\section{Parallelepipeds}
\label{sec:paralpyd}
\marginnote{Appears as problem 1 in L 06.}
We must find, among all square-base parallelepipeds with total surface $S$, the one
which minimizes the sum of all sides.
%
\begin{marginfigure}
    \includegraphics[width=1\linewidth]{qs_3_parapip_diag.tikz}
    \caption{Characteristics of the parallelepiped.}%
    \label{fig:pr3parapip}%
\end{marginfigure}
%

\paragraph{}
This problem is very short but absolutely non-trivial. That doesn't mean it is
very difficult; however, requires a good setup at the beginning.
As every minimization scenario, we must formulate the problem into an
equation. Let's start by writing down the formulas for the total surface
and the sum of all sides:
\begin{equation*}
    \begin{array}{cc}
        S = 2l^2 + 4lh & L = 8l + 4h \\
    \end{array}
\end{equation*}
Here, $l$ is the length of one side of the square base, $h$ is the height of
the solid, and $L$ is the sum of all sides.
Now let's try to think a little. We want to impose a certain surface $S$,
it is clear that we can play with $l$ and $h$ to generate many different
parallelepipeds that will keep $S$ unchanged. In this scenario, if we, for example,
try to increase $h$, we will see the solid getting taller, but the base will
probably shrink ($l$ will decrease) because we want to keep $S$ constant.
So let's take the equation for $S$, there are 3 variables therein: $S$, $l$ and $h$.
Let's consider $S$ fixed from now on; it means that $S$ will be a known value that we
choose. So now we have only 2 variables: $l$ and $h$: let's calculate one as function of
the other. We choose (of course) to
write $h = f(l)$\sidenote{Because the equation is linear in
$h$ and we can quickly isolate that term. If we chose to isolate $l$ instead,
we would need to solve a quadratic equation which is more difficult.}:
\begin{equation}\label{eq:hasfofl}
    h = \frac{S}{4} \cdot \frac{1}{l} - \frac{l}{2}
\end{equation}
Equation \ref{eq:hasfofl} sets a contraint on $h$ based on the values of $l$. If
we respect equation \ref{eq:hasfofl}, the values of $h$ and $l$ will cause
the total surface of the solid to be $S$.
Let's now substitute $h$ from equation \ref{eq:hasfofl} into the equation of
$L$:
\begin{equation}\label{eq:Lasf}
    L(l) = 8l + 4h = 8l + 4\left( \frac{S}{4} \cdot \frac{1}{l} - \frac{l}{2} \right)
        = \dots = 6l + \frac{S}{l}
\end{equation}
In equation \ref{eq:Lasf} $L$ is purposely indicated as a function of $l$. Now that we
have the total sum depending on $l$, we can find the extrema by following the normal
minimization process: compute the derivative, find its zeros and study the sign.
\begin{equation}\label{eq:Lfder}
    L^\prime(l) = \frac{d}{dl} \left( 6l + \frac{S}{l} \right) = 6 - \frac{S}{l^2}
\end{equation}
Let's impose $L^\prime(l) = 0$:
\begin{equation}\label{eq:Lroots}
    6 - \frac{S}{l^2} = 0 \implies \frac{6l^2 - S}{l^2} = 0 \implies
        6l^2 - S = 0 \implies l = \pm \sqrt{\frac{S}{6}}
\end{equation}
Value $\frac{S}{6}$ is positive so we can actually extract its square
root\sidenote{We must always check that all values make sense, this means
that equation \ref{eq:Lroots} is legit.}. We get 2 values
for $l$: one positive and one negative.
Since $l$ must be positive (it is a length), we must discard the negative one and
only accept $l = +\sqrt{\frac{S}{6}}$. But we still must make sure this value is
an extremum and a point of minimum.
To check that, we look at equation \ref{eq:Lfder}: $6l^2 - S$ is a parabola
pointing upwards\sidenote{The leading coefficient is $6 > 0$.}. The root
we have taken is the greatest one: it is where the parabola crosses the x-axis
from the vertex to infinity; therefore we pass from negative values to positive values,
hence we found a point of minimum. Now that we have $l$, we can calculate the minimum sum of all
sides $L^\ast$ to have surface $S$, by substituting in equation \ref{eq:Lasf}:
\begin{equation*}
    L^\ast =  6 \left( \frac{S}{6} \right)^{\frac{1}{2}} + S \left( \frac{S}{6} \right)^{-\frac{1}{2}}
\end{equation*}
Note that we must add $L^\ast > 0$
as a legit condition.

\section{Spheres}
\label{sec:speheres}
\marginnote{Appears as problem 1 in L 06.}
Something

\section{Tossing dices}
\label{sec:dices}
\marginnote{Appears as problem 1 in L 06.}
Something

\section{Magnetic fields and amperian loops}
\label{sec:magnamp}
\marginnote{Appears as problem 1 in L 06.}
Something

\section{Spaceships}
\label{sec:spaceship}
\marginnote{Appears as problem 1 in L 06.}
Something

\section{Protons in motion}
\label{sec:protons}
\marginnote{Appears as problem 1 in L 06.}
Something
