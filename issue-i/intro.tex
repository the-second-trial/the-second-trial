%%%%%%%%%%%%%%%%%%%%%%%%%%%%%%%%%%%%%%%%%%%%%%%%%%%%%%%%%%%%%%%%%%%%%%%%%%%%%%%
% The 2nd Trial by Andrea Tino is licensed under CC BY-NC-SA 4.0. To view a   %
% copy of this license, visit                                                 %
% http://creativecommons.org/licenses/by-nc-sa/4.0/                           %
%%%%%%%%%%%%%%%%%%%%%%%%%%%%%%%%%%%%%%%%%%%%%%%%%%%%%%%%%%%%%%%%%%%%%%%%%%%%%%%

\chapter*{Introduction}

This is the first issue of \TST and I think it would be best to start with a
few informatino about this work. Basically I want to address 3 simple
questions.

\paragraph{What?} \TST is a yearly released open publication that focuses
entirely on the second written part of the Italian
State Exam\footnote{Known in Italian as:
\textit{Esame di Stato} or, less informally: \textit{Esame di Maturita}.}:
the one about Mathematics. The focus of this book is to provide and
explain the solutions to the problems in the exam.

\paragraph{Who for?} The main target is students of course. I have designed
\TST specifically to be read by high school students who want to prepare
for the exam and need to exercise. However this is also a good tool for
Math teachers for preparing training sessions with their students.\\
If this is about Italy, why English? I think about \TST as learning
material; my idea is to allow students to also familiarize with
English expressions and lingo that are used by mathematicians.

\paragraph{Why?} My aim is not to simply provide solutions to the
different problems in the exam, but also to have fun with it.
When focusing on the problems, we will try to find different paths to
solutions, including the use of computer algorithms and coding.

\paragraph{} Now let's get down to business and talk about the problems
that were put inside the 2019 edition of the exam. This year the Ministry
has decided to focus on Physics and Magnetism as driving theme. We find,
in fact, a whole problem about magnetic fields, and some of the questions
include topics related to Physics.
