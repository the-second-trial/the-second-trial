%%%%%%%%%%%%%%%%%%%%%%%%%%%%%%%%%%%%%%%%%%%%%%%%%%%%%%%%%%%%%%%%%%%%%%%%%%%%%%%
% The 2nd Trial by Andrea Tino is licensed under CC BY-NC-SA 4.0. To view a   %
% copy of this license, visit                                                 %
% http://creativecommons.org/licenses/by-nc-sa/4.0/                           %
%%%%%%%%%%%%%%%%%%%%%%%%%%%%%%%%%%%%%%%%%%%%%%%%%%%%%%%%%%%%%%%%%%%%%%%%%%%%%%%

\documentclass[twoside,symmetric]{tufte-book}

\hypersetup{colorlinks} % colored hyperlinks (e.g., for onscreen viewing)

%%
% Book metadata
\title{The 2nd Trial}
\author[Andrea Tino]{Andrea Tino}
\publisher{Issue I - Year 2019}

%\usepackage{microtype}

%%
% For nicely typeset tabular material
\usepackage{booktabs}

%%
% For graphics / images
\usepackage{graphicx}
\setkeys{Gin}{width=\linewidth,totalheight=\textheight,keepaspectratio}
\graphicspath{{graphics/}}

% The fancyvrb package lets us customize the formatting of verbatim
% environments.  We use a slightly smaller font.
\usepackage{fancyvrb}
\fvset{fontsize=\normalsize}

%%
% Prints argument within hanging parentheses (i.e., parentheses that take
% up no horizontal space).  Useful in tabular environments.
\newcommand{\hangp}[1]{\makebox[0pt][r]{(}#1\makebox[0pt][l]{)}}

%%
% Prints an asterisk that takes up no horizontal space.
% Useful in tabular environments.
\newcommand{\hangstar}{\makebox[0pt][l]{*}}

%%
% Prints a trailing space in a smart way.
\usepackage{xspace}

% Prints the month name (e.g., January) and the year (e.g., 2008)
\newcommand{\monthyear}{%
  \ifcase\month\or January\or February\or March\or April\or May\or June\or
  July\or August\or September\or October\or November\or
  December\fi\space\number\year
}


% Prints an epigraph and speaker in sans serif, all-caps type.
\newcommand{\openepigraph}[2]{%
  %\sffamily\fontsize{14}{16}\selectfont
  \begin{fullwidth}
  \sffamily\large
  \begin{doublespace}
  \noindent\allcaps{#1}\\% epigraph
  \noindent\allcaps{#2}% author
  \end{doublespace}
  \end{fullwidth}
}

% Inserts a blank page
\newcommand{\blankpage}{\newpage\hbox{}\thispagestyle{empty}\newpage}

\usepackage{units}

% Typesets the font size, leading, and measure in the form of 10/12x26 pc.
\newcommand{\measure}[3]{#1/#2$\times$\unit[#3]{pc}}

% Macros for typesetting the documentation
\newcommand{\hlred}[1]{\textcolor{Maroon}{#1}}% prints in red
\newcommand{\hangleft}[1]{\makebox[0pt][r]{#1}}
\newcommand{\hairsp}{\hspace{1pt}}% hair space
\newcommand{\hquad}{\hskip0.5em\relax}% half quad space
\newcommand{\TODO}{\textcolor{red}{\bf TODO!}\xspace}
\newcommand{\ie}{\textit{i.\hairsp{}e.}\xspace}
\newcommand{\eg}{\textit{e.\hairsp{}g.}\xspace}
\newcommand{\na}{\quad--}% used in tables for N/A cells
\providecommand{\XeLaTeX}{X\lower.5ex\hbox{\kern-0.15em\reflectbox{E}}\kern-0.1em\LaTeX}
\newcommand{\tXeLaTeX}{\XeLaTeX\index{XeLaTeX@\protect\XeLaTeX}}
% \index{\texttt{\textbackslash xyz}@\hangleft{\texttt{\textbackslash}}\texttt{xyz}}
\newcommand{\tuftebs}{\symbol{'134}}% a backslash in tt type in OT1/T1
\newcommand{\doccmdnoindex}[2][]{\texttt{\tuftebs#2}}% command name -- adds backslash automatically (and doesn't add cmd to the index)
\newcommand{\doccmddef}[2][]{%
  \hlred{\texttt{\tuftebs#2}}\label{cmd:#2}%
  \ifthenelse{\isempty{#1}}%
    {% add the command to the index
      \index{#2 command@\protect\hangleft{\texttt{\tuftebs}}\texttt{#2}}% command name
    }%
    {% add the command and package to the index
      \index{#2 command@\protect\hangleft{\texttt{\tuftebs}}\texttt{#2} (\texttt{#1} package)}% command name
      \index{#1 package@\texttt{#1} package}\index{packages!#1@\texttt{#1}}% package name
    }%
}% command name -- adds backslash automatically
\newcommand{\doccmd}[2][]{%
  \texttt{\tuftebs#2}%
  \ifthenelse{\isempty{#1}}%
    {% add the command to the index
      \index{#2 command@\protect\hangleft{\texttt{\tuftebs}}\texttt{#2}}% command name
    }%
    {% add the command and package to the index
      \index{#2 command@\protect\hangleft{\texttt{\tuftebs}}\texttt{#2} (\texttt{#1} package)}% command name
      \index{#1 package@\texttt{#1} package}\index{packages!#1@\texttt{#1}}% package name
    }%
}% command name -- adds backslash automatically
\newcommand{\docopt}[1]{\ensuremath{\langle}\textrm{\textit{#1}}\ensuremath{\rangle}}% optional command argument
\newcommand{\docarg}[1]{\textrm{\textit{#1}}}% (required) command argument
\newenvironment{docspec}{\begin{quotation}\ttfamily\parskip0pt\parindent0pt\ignorespaces}{\end{quotation}}% command specification environment
\newcommand{\docenv}[1]{\texttt{#1}\index{#1 environment@\texttt{#1} environment}\index{environments!#1@\texttt{#1}}}% environment name
\newcommand{\docenvdef}[1]{\hlred{\texttt{#1}}\label{env:#1}\index{#1 environment@\texttt{#1} environment}\index{environments!#1@\texttt{#1}}}% environment name
\newcommand{\docpkg}[1]{\texttt{#1}\index{#1 package@\texttt{#1} package}\index{packages!#1@\texttt{#1}}}% package name
\newcommand{\doccls}[1]{\texttt{#1}}% document class name
\newcommand{\docclsopt}[1]{\texttt{#1}\index{#1 class option@\texttt{#1} class option}\index{class options!#1@\texttt{#1}}}% document class option name
\newcommand{\docclsoptdef}[1]{\hlred{\texttt{#1}}\label{clsopt:#1}\index{#1 class option@\texttt{#1} class option}\index{class options!#1@\texttt{#1}}}% document class option name defined
\newcommand{\docmsg}[2]{\bigskip\begin{fullwidth}\noindent\ttfamily#1\end{fullwidth}\medskip\par\noindent#2}
\newcommand{\docfilehook}[2]{\texttt{#1}\index{file hooks!#2}\index{#1@\texttt{#1}}}
\newcommand{\doccounter}[1]{\texttt{#1}\index{#1 counter@\texttt{#1} counter}}

% Generates the index
\usepackage{makeidx}
\makeindex

\begin{document}

% Front matter
\frontmatter

% Blank page before title
%\blankpage

% Full title page
\maketitle

% Copyright page
\newpage
\begin{fullwidth}
~\vfill
\thispagestyle{empty}
\setlength{\parindent}{0pt}
\setlength{\parskip}{\baselineskip}
The 2nd Trial by \thanklessauthor

\par\smallcaps{Series: \thanklesspublisher}

\par\smallcaps{github.com/andry-tino/the-second-trial}

\par The 2nd Trial by Andrea Tino is licensed under
Attribution-NonCommercial-ShareAlike 4.0 International.
To view a copy of this license, visit
\url{http://creativecommons.org/licenses/by-nc-sa/4.0/}.\index{license}

\par\textit{First printing, \monthyear}
\end{fullwidth}

% Contents
\tableofcontents
%\listoffigures
%\listoftables

% Introduction
\cleardoublepage
\chapter*{Introduction}

This sample book discusses the design of Edward Tufte's
books\cite{Tufte2001,Tufte1990,Tufte1997,Tufte2006}
and the use of the \doccls{tufte-book} and \doccls{tufte-handout} document classes.



%%
% Start the main matter (normal chapters)
\mainmatter


\chapter[On the Use of the tufte-book Document Class]{On the Use of the \texttt{tufte-book} Document Class}
\label{ch:tufte-book}

The document classes define a style similar to the
style Edward Tufte uses in his books and handouts.  Tufte's style is known
for its extensive use of sidenotes, tight integration of graphics with
text, and well-set typography.  This document aims to be at once a
demonstration of the features of the ocument classes
and a style guide to their use.

\section{Page Layout}\label{sec:page-layout}
\subsection{Headings}\label{sec:headings}\index{headings}
This style provides \textsc{a}- and \textsc{b}-heads (that is,
\Verb|\section| and \Verb|\subsection|), demonstrated above.

If you need more than two levels of section headings, you'll have to define
them yourself at the moment; there are no pre-defined styles for anything below
a \Verb|\subsection|.  As Bringhurst points out in \textit{The Elements of
Typographic Style},\cite{Bringhurst2005} you should ``use as many levels of
headings as you need: no more, and no fewer.''

The lasses will emit an error if you try to use
\linebreak\Verb|\subsubsection| and smaller headings.

% let's start a new thought -- a new section
\newthought{In his later books},\cite{Tufte2006} Tufte
starts each section with a bit of vertical space, a non-indented paragraph,
and sets the first few words of the sentence in \textsc{small caps}.  To
accomplish this using this style, use the \doccmddef{newthought} command:
\begin{docspec}
  \doccmd{newthought}\{In his later books\}, Tufte starts\ldots
\end{docspec}


\section{Sidenotes}\label{sec:sidenotes}
One of the most prominent and distinctive features of this style is the
extensive use of sidenotes.  There is a wide margin to provide ample room
for sidenotes and small figures.  Any \doccmd{footnote}s will automatically
be converted to sidenotes.\footnote{This is a sidenote that was entered
using the \texttt{\textbackslash footnote} command.}






%%
% The back matter



\backmatter

\bibliography{biblio}
\bibliographystyle{plainnat}


\printindex

\end{document}

